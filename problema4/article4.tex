\documentclass{article}
\usepackage{algpseudocode,algorithm,algorithmicx}
\usepackage[usenames,dvipsnames,svgnames,table]{xcolor}

\begin{document}
    \title{\textcolor{blue}{\textbf{Problema 4 }}\\}
    \author{Daniel de la Cruz Prieto C211\\ David Orlando De Quesada Oliva C211\\Javier Dominguez C212} 
    \date{}
    \maketitle  

    \section*{Orden del problema} 
    Sea $G = <V, E>$ un grafo no dirigido y conexo, se desea saber si G cumple la propiedad de que cada par de
    v\'ertices existen dos caminos disjuntos en v\'ertices entre ellos. La complejidad temporal de su algoritmo debe
    ser $O(|V| + |E|)$.
    \newline
    \newline
    Lema1:\newline
    $\forall x,y\in V(G)$ existen dos caminos disjuntos en v\'ertices $\Longleftrightarrow$ G es biconexo \newline
    \newline
    Demostraci\'on:\newline
    $\Longrightarrow$\newline
    Sea x un v\'ertice arbitrario de G. En el grafo G-x(resultado de remover el v\'ertice x del grafo G)
    sigue habiendo un camino entre cualquier par de v\'ertices u,v pues como en G existen dos caminos disjuntos en v\'ertices
    entre u y v si x pertenec\'ia a uno de estos dos caminos no importa pues al removerlo sigue habiendo al menos un camino 
    entre u y v por lo que G-x ser\'ia conexo. Entonces al remover cualquier v\'ertice de G y todas las aristas incidentes a
    \'el no desconectan el grafo . Luego G es biconexo.\newline
    $\Longleftarrow$\newline
    Sea G un grafo biconexo y u,v v\'ertices del mismo.Vamos a demostrar por inducci\'on en la distancia $d(u,v)$ que hay
    dos caminos disjuntos en v\'ertices de u a v $\forall u,v \in V(G)$.
    \newline
    Caso Base $d(u,v)=1$:\newline
    Si $d(u,v)=1$ entonces la arista $<u,v>$ es uno de los caminos que conectan a u con v.Como G es biconexo entonces 
    G-u y G-v son conexos y por tanto si quitas la  arista $<u,v>$  como el grafo no se desconecta sigue existiendo
    un camino que conecta a u con v y que no contiene a $<u,v>$ ,este ser\'ia nuestro segundo camino.\newline
    
    Paso Inductivo:\newline
    Sea G un grafo donde $\forall\; u,v\in V(G)\; d(u,v)=k>1$. Entonces en G existe un camino de longitud k entre 
    cualquier par de v\'ertices u,v.Sea z el \'ultimo v\'ertice anterior a v en el camino de de longitud k que va de
    u a v .Obtenemos $d(u,z)=d(u,v)-1=k-1$ . Por hip\'otesis de inducci\'on existen dos caminos disjuntos en v\'ertices 
    $P_{1},P_{2}$ de u a v.Dado que G-z es conexo pues G es biconexo ,entonces existe un camino $P_{q}$ que va de u a v
    y no contiene a z.Si $P_{q}$ no intercepta con $P_{1}$ o con $P_{2}$,tenemos los dos caminos disjuntos en v\'ertices
    de u a v:$P_{q}$ y $P_{1}$ o $P_{2}$(uno de los 2 con el que no tiene intercepci\'on) + $<u,v>$.Si $P_{q}$ intercepta 
    con $P_{1}$ y con $P_{2}$ vamos a crear un camino que comienza en v sigue por Q hasta la primera intercepci\'on con
    $P_{1}$ o $P_{2}$ y sigue por uno de estos dos hasta u,este camino no contiene a z. Ahora crea que otro camino que vaya 
    desde u por $P_{q}$ hasta la primera intercepci\'on con $P_{1}$ o $P_{2}$ ,ahora sigue por $P_{1}$ o $P_{2}$ hasta z 
    y por la arista $<z,v>$ llega a v.Encontr\'e dos caminos disjuntos en v\'ertices que van de u a v.\newline
    Luego en G hay dos caminos disjuntos en v\'ertices de u a v $\forall u,v \in V(G)$
    \newline
    \newline
    Lema2:\newline
    G es biconexo $\Longleftrightarrow$ G no tiene puntos de articulaci\'on \newline
    Demostraci\'on:\newline
    $\Longleftarrow$\newline
    Si G es biconexo entonces al quitar cualquier v\'ertice v de G y las aristas incidentes al mismo 
    G sigue siendo conexo por lo no se divide en 2 o m\'as de componentes conexas .Por tanto G no
    tiene puntos de articulaci\'on.\newline
    $\Longleftarrow$\newline
    Si G no tiene puntos de articulaci\'on entonces no existe ning\'un  v\'ertice v que quite de G que lo separe en
    2 o m\'a  componentes conexas.Entonces G-v sigue siendo conexo.Luego G es biconexo.
    \newline
    \newline
    Lema3:\newline
    $\forall x,y\in V(G)$ existen dos caminos disjuntos en v\'ertices $\Longleftrightarrow$ G no tiene puntos de articulaci\'on\newline
    Demostraci\'on:\newline
    Por Lema1  $\forall x,y\in V(G)$ existen dos caminos disjuntos en v\'ertices $\Longleftrightarrow$ G es biconexo. 
    Por Lema2 G es biconexo $\Longleftrightarrow$ G no tiene puntos de articulaci\'on.
    Entonces por transitividad $\forall x,y\in V(G)$ existen dos caminos disjuntos en v\'ertices 
    $\Longleftrightarrow$ G no tiene puntos de articulaci\'on
    \newline
    \newline
    \newline
    Correctitud del problema:
    \newline
    Por lo expuesto con anterioridad en los lemas 1,2 y 3 y por la correctitud del algoritmo de detecci\'on de puntos de 
    articulaci\'on  podemos asegurar la correctitud de nuestro algoritmo para detectar si  un grafo G cumple la propiedad 
    de que cada par de v\'ertices existen dos caminos disjuntos en v\'ertices entre ellos.
    \newline
    \newline
    \newline
    Complejidad Temporal:\newline
    La complejidad temporal de nuestro algoritmo es la complejidad temporal del algoritmo de detecci\'on de puntos 
    de articulaci\'on que es $O(|V|+|E|)$
    \newline
    \newline
    Pseudoc\'odigo:
    
    \begin{algorithm}
        \caption{Determinar si un grafo G cumple $\forall x,y\in V(G)$  existen dos caminos disjuntos en v\'ertices de x a y }
        \textbf{Solve($G,x,y,w$)\\}
        1-\hspace*{1em}DFS-Visit-PA($G,u$) \\ 
        2-\hspace*{2em}u $\leftarrow$ visited \\
        3-\hspace*{2em}$time=time+1$ \\
        4-\hspace*{2em}$d[u]=tome$\\
        5-\hspace*{2em}$low[u]=d[u]$\\ 
        6-\hspace*{2em}for each v $\in$ $Adj[u]$\\
        7-\hspace*{3em}do if v not visited\\
        9-\hspace*{4em}$\pi[v] \leftarrow u$\\
        10-\hspace*{4em}DFS-VISIT-PA(G,v)\\
        11-\hspace*{4em}$low[u]=min(low[u],low[v])$\\
        12-\hspace*{4em}if $low[v]\ge d[u]$\\
        13-\hspace*{5em}return False\\
        14-\hspace*{3em}else if $pi[u]\ne v$\\
        15-\hspace*{4em}$low[u]=min(low[u],d[v])$\\ 
        15-\hspace*{2em}return True\\\\
        17-\hspace*{1em}DFS-PA($G$)\\
        18-\hspace*{2em}for each v $\in$ V(G)\\
        19-\hspace*{3em}do if not visited\\
        20-\hspace*{4em}if $deg(v)\ge 2$\\
        21-\hspace*{5em}return False\\
        22-\hspace*{4em}if not DFS-Visit-PA(G,v)\\
        23-\hspace*{5em}return False\\
        24-\hspace*{2em}return True\\---No encont\'e ningu punto de articulaci\'on y por lo tanto no se cumple que para todo par de v\'ertices existen dos caminos disjuntos en v\'ertices
        %Me parece que seria mejor si lo agrego a una lista y despues pregunto por lista.count para no modificar mucho del codigo
    \end{algorithm}

\end{document}
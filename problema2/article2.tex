\documentclass{article}
\usepackage{hyperref}
\usepackage{algpseudocode,algorithm,algorithmicx}
\usepackage[usenames,dvipsnames,svgnames,table]{xcolor}

\begin{document}
    \title{\textcolor{blue}{\textbf{Problema 2}}\\}
    \author{Daniel de la Cruz Prieto C211\\ David Orlando De Quesada Oliva C211\\Javier Dominguez C212} 
    \date{}
    \maketitle  

    \section{\underline{Orden del Problema}}
    
    Disene un algoritmo que dado un grafo no dirigido con costo en las aristas $G = <V, E>$ y los v\'ertices x, $y \in  V$ ,
    determine el costo de la arista de menor costo entre las aristas de mayor costo en cada camino de x a y. Esto
    quiere decir que si de x a y existen dos caminos $P_1$ , $P_2$ y la arista de mayor costo en $P_1$ es 5 y la de mayor costo
    en $P_2$ es 4, entonces deben devolver 4. La complejidad temporal de su algoritmo debe ser de $O(|E|log(|V|))$.\\

    \section{\underline{Soluci\'on}}

    Proposici\'on 1 : \\
    
    Sea $G = <V,E>$ grafo no dirigido con costo en las aristas, $x$, $y$ $\in V(G)$, $y$ alcanzable desde $x$, $p_1, p_2,..., p_t$ 
    los caminos de costo m\'inimo de $x$ a $y$ en $G$, $e_1, e_2,..., e_t$ las aristas de costo m\'aximo en cada uno de los caminos $p_i$, $1 \leq i \leq t$, y
    $E'$ = \textbraceleft $e \in E(G) |\hspace{0.2em} w(e)$ = min \textbraceleft $w(e_i) |\hspace{0.2em} 1 \leq i \leq t$ \textbraceright \textbraceright, el conjunto
    de las aristas de menor costo entre las $e_i$. Entonces el algoritmo de PRIM tomando como v\'ertice inicial $x$, devuelve un AACM T de G, tal que $\exists\hspace{0.2em}
    e' \in E', e' \in E(T)$ y el \'unico camino de $x$ a $y$ en el AACM T pasa por $e'$.\\\\
    
    Inicialmente, como es\'a descrito en conferencia en la cola con prioridad que mantiene el corte(o sea el conjunto de v\'ertices a\'un no visitados por el algoritmo)est\'an todos los 
    v\'ertices del grafo G. En cada paso del algoritmo se a\~nade una nueva arista de forma tal que siempre se mantiene un conjunto de aristas que son un subcojunto de alg\'un AACM T de G,
    como se describe en la conferencia. Por esto, en alg\'un momento del algoritmo $\exists\hspace{0.2em} e'' = \hspace{0.2em}<u,v>\hspace{0.2em} \in E'$ que ser\'a la arista candidata a 
    a\~nadir al conjunto de aristas que se tiene hasta el momento, y puede ocurrir dos cosas :\\\\
    
    $\bullet$ Que se a\~nada dicha arista, o sea, que el v\'ertice $v$ a\'un no haya sido alcanzado por el algortimo usando alg\'un otro camino de $x$ a $y$. En este caso ya tenemos que cuando 
    se termine de ejecutar el algoritmo se cumplir\'a que $\exists e' \in E', e' \in E(T)$ donde $T$ es el AACM resultante al terminar el algoritmo de PRIM.\\\\

    $\bullet$ Que no se a\~nada dicha arista, o sea, que el v\'ertice $v$ haya sido alcanzado por el algoritmo, usando otro de los caminos que van de $x$ a $y$.
    Digamos q ese camino fue $p_k$, entonces $e_k$ no puede tener costo mayor que $e''$, pues, en dicho caso, el propio algortimo garantiza que se tome la arista de menor costo
    de todas las que cruzan el corte y en dicho caso ser\'ia $e''$, por lo tanto $v$ no puede haber sido alcanzado usando $p_k$, lo que es una contradicci\'on pues asumimos que $v$
    fue alcanzado usando $p_k$. Por tanto la \'unica posiblidad es que $e_k \in E'$, o sea $e_k$ es una de las aristas que tiene costo m\'inimo entre todas las aristas
    de mayor costo en cada camino de $x$ a $y$ ($w(e_k) = w(e''))$, y por tanto se cumple que $e_k \in E', e_k \in E(T)$ donde $T$ es el AACM resultante al terminar el algoritmo de PRIM.\\\\

    $\Rightarrow$ Por tanto al finalizar el algoritmo de PRIM, ejecutado sobre  el grafo G, iniciando en el v\'ertice x, se cumple que 
    $\exists e' \in E', e' \in E(T)$, siendo $T$ el AACM resultante al terminar el algoritmo de PRIM.\\\\

    Sea $e' = <u,v>$, hay un \'unico camino de $x$ a $u$ en $T$, y hay un \'unico camino de $v$ a $y$ en $T$, y como consecuencia de que esta arista est\'a en todo AACM $T$ de $G$
    $\Rightarrow$ hay un \'unico camino de $x$ a $y$ en $T$ que pasa por la arista $e'$\\\\

    $\bullet$ Por tanto se cumple la proposici\'on 1\\\\
    \newpage
    \section{\underline{Pseudoc\'odigo}}
    \begin{algorithm}
        \caption{Determinar el costo de la arista de menor costo entre las aristas de mayor costo en cada camino de $x$ a $y$ en un grafo $G$ con funci\'on de costo $w$}
        \textbf{Solve($G,x,y,w$)\\} 
        1-\hspace*{1em}$T$ $\leftarrow$ PRIM($G,x,w$) --------- Construir un AACM de $G$ empezando en $x$. $O(|E|\log{|V|})$\\
        2-\hspace*{1em} if $\pi$[y] != NULL:\\
        3-\hspace*{2em}start $\leftarrow y$, answer $\leftarrow$ $w(\pi[y],y)$. $O(1)$\\ 
        4-\hspace*{2em}while start != $x$ do:  $O(|V|)$\\
        5-\hspace*{3em}start $\leftarrow$ $\pi$[start] $O(1)$\\
        6-\hspace*{3em}answer $\leftarrow$ m\'ax(answer,w($\pi$[start],start)) $O(1)$\\
        7-\hspace*{2em} return answer\\\\
        
        l\'ineas 3-7 Quedarse con el mayor de los costos de las aristas en el camino de $x$ a $y$ en $T$
    \end{algorithm}

    \section{\underline{Correctitud del Algoritmo}}

    La correctitud del algoritmo se basa principalmente en la de los algoritmos de DFS y PRIM vistos en conferencia, y en esencia en 
    la aplicaci\'on de la proposici\'on 1 previamente demostrada, una vez verificado que $y$ es alcanzable desde $x$, utilizando el array 
    $\pi$, que computa el propio algoritmo de PRIM al construir el AACM empezando en x, solo resta quedarse con la arista de mayor costo 
    en el camino de $x$ a $y$ en $T$, operaci\'on que realizan las l\'ineas 5-9 del pseudoc\'odigo usando el array $\pi$.

    \section{\underline{Complejidad Temporal}}
    La complejidad temporal del algoritmo es:\\ $O(|E|\log{|V|}) + O(|V|) = O(max(|E|\log{|V|}, |V|)) = O(|E|\log{|V|})$
\end{document}
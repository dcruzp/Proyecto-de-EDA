\documentclass{article}


\usepackage{algpseudocode,algorithm,algorithmicx}
\usepackage[usenames,dvipsnames,svgnames,table]{xcolor}
\usepackage{hyperref}
\usepackage{amsmath}

\begin{document}
    \title{\textcolor{blue}{\textbf{Problema 3 }}\\}
    \author{Daniel de la Cruz Prieto C211\\ David Orlando De Quesada Oliva C211\\Javier Dominguez C212} 
    \date{}
    \maketitle  

    \section{Descripci\'on del problema} 

    Sea $G = <V,E>$ un grafo no dirigido y conexo con pesos positivos en las aristas, y un v\'ertice $s \in V$ . Un \'arbol de 
    caminos de costo m\'inimos con ra\'iz en s es un subgrafo $T = <V^{'} , E^{'}> $ donde $V^{'} \subseteq  V $     y $E^{'} \subseteq  E $ tal que: 
    
    \begin{itemize}
        \item $V^{'}$ es el conjunto de v\'ertices alcanzables desde s en $G$
        \item $T$ es un \'arbol con ra\'iz en $s$
        \item Para todo $v \in V^{'}$ el \'unico camino simple de $s$ a $v$ en $T$ es un camino de costo m\'inimo de $s$ a $v$ en $G$ .
    \end{itemize}

    \noindent Dise\~ne un algoritmo que encuentre en el grafo G un \'arbol T de caminos de costo m\'inimo con ra\'iz en s tal que
    la suma de los pesos de las aristas de T sea lo menor posible. La complejidad temporal del algoritmo debe ser
    $O(|E|log|V|)$.


    \section{ Explicaci\'on de la soluci\'on }

    \noindent La soluci\'on al problema es aplicar el algoritmo de  Dijkstra visto en conferencias.
    Para encontrar un subgrafo $G^{'}$ que contenga todas las aristas que forman parte de algun camino de 
    costo minimo desde $s$ hasta cualquier otro v\'ertice del grafo $G$. Luego encontramos un \'arbol $T$ abarcador de 
    costo m\'inimo en $G^{'}$ , $T$ es el \'arbol que estamos buscando 


    \section{Demostraci\'on  de la correctitud del algoritmo}

    \noindent La correctitud del algoritmo se basa en la correctitud del algoritmo de Dijkstra visto en conferencia 
    y en la correctitud del algorimo de Prim visto en Conferencia tambi\'en. Ademas hay que demostrar que el \'arbol $T$ 
    devuelto por el algoritmo es un \'arbol de caminos de costo m\'inimo y que la suma del peso de sus aristas es la menor posible  
    
    \subsection*{Proposiciones }
    \begin{itemize}
        \item Vamos a denotar $\delta\left(u,v\right)$ como la distancia m\'inima de $u$ a $v$ 
        \item vamos a denotar a $w(u,v)$ como el peso de la arista $e= <u,v>$ 
        \item Sea $G = <V,E>$ un Grafo  vamos a denotar a $G' = <V',E'>$ como un subgrafo de $G$ que solo contiene las aristas $ u \xrightarrow[]{e} v$ 
        que cumplen que $\delta\left(s,u\right) + w \left(u,v\right) = \delta\left(s,v\right)$   
    \end{itemize}

    \subsection*{Lema 1:}

    \textit{Sean u,v v\'ertices de G .Si el costo de las aristas es positivo todo camino de costo m\'inimo de u a v es un camino simple de u a v.}
    
    \vspace*{0.3cm} 

    \noindent \textbf{Demostraci\'on:}
    \\*
    Si $p_{m}$un camino de costo m\'inimo de u a v no fuera simple es porque tiene al menos un ciclo c .Como todas las aristas que participan en ese
    camino tienen costo positivo quita el ciclo c y tendr\'ias un camino de u a v cuyo costo ser\'ia m\'as chico que el costo de $p_{m}$ lo cual generar
    una contradicci\'on pues $p_{m}$ es un camino de costo m\'inimo de u a v.\\
    
    \subsection*{Lema 2:}
        
    \noindent \textit{Sea $P = \left(v_0,v_1,\dots , v_k \right)$ con $v_0 = s$ y $v_k = v$ un camino simple de $s$ a $v$. Entonces $P$ es un 
            camino de costo m\'inimo de $s$ a $v$  en $G$ $\Longleftrightarrow$  $P$ es un camino de $s$ a $v$ $G'$}
    
    
    \vspace{0.3cm}
    \noindent \textbf{Demostraci\'on} 
    
    \noindent $ \Longleftarrow $
    \\*
    Sea $P = \left(v_0,v_1,\dots , v_k \right)$  un camino en $G'$ . 
    Vomos a hacer la demostraci\'on por inducci\'on en la cantidad de aristas del camino $P$ (vamos a denotar a $k$ como la cantidad de aristas en el camino $P$ )
    \\*
    Para el caso base tenemos que si $k=0$ entonces el camino $P$ no tiene aristas por lo que es un camino de costo m\'inimo en $G$ 
    \\*
    Para el paso inductivo vamos a suponer que se cumple que para todo camino de $k-1$ aristas que pertenece a  $G'$ es un camino de costo m\'inimo en $G$. 
    Entonces vamos a suponer que tenemos un camino $P = \left(v_{0},\dots,v_{k}\right)$  de $k$ aristas en $G'$. En particular tenemos que el subcamino $\left(v_0,\dots,v_{k-1}\right)$ es un camino de 
    costo m\'inimo en $G$ (por hip\'otesis de inducci\'on ). Ahora si $P$ no fuera un camino de costo m\'inimo entonces: 
    \begin{equation*}
        \delta \left(s,v_k\right) < \sum_{i=1}^{k} w\left(v_{i-1,v_i}\right) = \sum_{i=1}^{k-1} w\left(v_{i-1}, v_{i}\right) + w\left(v_{k-1}, v_{k}\right)  = \delta\left(s,v_{k-1}\right) + w\left(v_{k-1}, v_{k}\right)
    \end{equation*}

    \noindent esto contradice que $<v_{k-1}, v_{k}> \in E'$ . Entonces un camino $P$  en $G'$ que tiene $k$ aristas es un camino de costo m\'inimo en $G$ 
    \\
    Luego todo camino $P = \left(v_0,v_1,\dots , v_k \right)$ donde $v_{0}=s$ y $v_{k}=v$ de $G'$ es un camino  de costo m\'inimo en $G$ de $s$ a $v$ 
    
    \vspace*{0.5cm} 

    \noindent $\Longrightarrow $
    \\*
    Ahora sea $P = \left(v_0,\dots,v_k \right)$ un camino de costo m\'inimo de $v_0 = s $ a  $v_k = v$ en $G$ que tiene $k$ aristas (la demostraci\'on la vamos a hacer por inducci\'on en $k$).
    \\*
    Si $k=0$ entonces el camino no contine aristas y por lo tanto es un camino en $G'$ que comienza en $s$ .
    \\*
    ahora en el paso inductivo tenemos para $k>0$ .Si todos los caminos de costo m\'inimo que comienzan en $s$  que tienen $k-1$ aristas en $G$ corresponden a un camino que comienza en $s$ en $G'$ (por el paso inductivo), entonces en particular  $\left(v_0,\dots,v_{k-1}\right)$
    pertenece a $G'$ , esto quiere decir que $<v_{i-1},v_i> \in E'$ para toda $i \in \{1,2,\dots, k-1\}$. Si el camino hasta $v$ no pertenes a $G'$ entoces esto solo puede suceder porque la arista  $<v_{k-1}, v_{k}> \notin E'$ .Entonces tenemos:
    \begin{equation*}
        \sum_{i=1}^{k} w\left(v_{i-1},v_{i}\right) = \sum_{i=1}^{k-1} w\left(v_{i-1},v_{i}\right) + w\left(v_{k-1} , v_{k}\right) = \delta\left(s,v_{k-1}\right) + w\left(v_{k-1, v}\right) > \delta\left(s, v\right)
    \end{equation*}
    \noindent Esto contradice lo que habiamos supuesto , que $\left(v_0,v_1.\dots, v_k\right)$ era un camino de costo m\'inimo en $G$ , luego queda demostrado que para un k arbitrario se cumple. (por inducci\'on )
    \\*
    Luego tenemos que un camino de costo m\'inimo de $s$ a $v$ en $G$ corresponde a un camino de $s$ a $v$ en $G'$ 
    
    \vspace*{0.5cm} 

    \subsection*{Lema 3:}
    
    \textit{En el grafo $G'$ $\forall v \in V(G) $ con $v\neq s$  est\'an todos los caminos de costo m\'inimo de $s$ a $v$ que hay en $G$}
    
    \vspace*{0.3cm} 
     
    \noindent \textbf{Demostraci\'on:}
    \\*
    Como por el Lema1 todo  camino de costo m\'inimo de G es un camino simple y por el lema 2 todo camino simple de
    costo m\'inimo  de G est\'a en $G'$ entonces en $G'$ est\'an  todos los caminos de costo m\'inimo de desde s hacia 
    cualquier otro v\'ertice de G .Luego en $G'$  est\'an todos los caminos de costo m\'inmo de s a v. 
    $\forall v\in V(G)$

    \vspace*{0.3cm} 

    \noindent  Por el Lema3 tenemos que todo \'arbol de caminos de costo m\'inimo que esta en $G$ tambi\'en esta en $G'$  y viceversa.\\
    Luego como el algoritmo de Prim nos
    
    \noindent \textbf{Proposici\'on:}
    \\*
    Un AACT T de $G'$ con ra\'iz s cumple lo siguiente:\\
    1-T un \'arbol con ra\'iz en s\\
    2-Todo v\'ertice alcanzable por s en G est\'a en $G'$\\
    3-Todo $\forall v$ si v es alcanzable por s entonces el camino simple u\'nico de s a v es un camino de costo m\'inimo\\
    4-T es un arb\'ol de caminos de costo mı́nimo tal que la suma de los pesos de las aristas de T sea lo menor posible.\\
    % \item 
    devuelve el \'arbol abarcador de costo m\'inimo tomando como ra\'iz el v\'ertice $s$ . el cual se aplica sobre el grafo $G'$ donde todo camino desde $s$ es un camino de  costo m\'inimo.
    Entonces tenemos que el \'arbol abarcador devuelto por el algoritmo de Prim es un \'arbol de caminos de costo m\'inimo ,(porque se aplica sobre el grafo $G'$ ) y adem\'as la suma de las aristas es la 
    menor posible, por ser un \'arbol abarcador de costo m\'inimo de $G'$ . 
    \\*
    Ahora por el teorema enunciado arriba podemos afirmar que el algoritmo devuelve el \'arbol de  caminos m\'inimos con ra\'iz en $s$ tal que la suma del peso de sus aristas es la menor posible . 
    \\
    En generar la correctitud esta dada por el algoritmo de Dijkstra (que utilizamos para deteminar $G'$), la correctitud del algoritmo de Prim (que lo utilizamos para saber cual es el \'arbol de costo m\'inimo de $G'$) y el Teorema 1 el cual nos 
    garantiza que todo \'arbol de caminos m\'inimo  en $G$ esta tambi\'en en $G'$ y viceversa . 

    \section{Pseudoc\'odigo del Algoritmo }

    \begin{algorithm}[H]
        \caption{Determinar el \'arbol de caminos m\'inimos desde $s$ de menor peso }
        %\begin{algorithmic}[1] 
        %    \State $d[\hspace*{0.1cm}]$ $\leftarrow $ Dijkstra $\left(G,s\right)$   
        %    \State $E (G^{'})$ $\leftarrow$ $\emptyset$
        %    \State $V(G')$ $\leftarrow$ $V(G)$
        %    \ForAll {$u \xrightarrow[]{e} v$  $\in$ $E(G)$}
        %        \If {$d[u] + w(e) = d[v] $} 
        %            \State $E(G^{'}) = E(G^{'})  \cup  e $ 
        %        \EndIf
        %    \EndFor
        %    \State $T \leftarrow  PrimAlgorithm (G^{'} , s)$
        %    \State return $T$ 
        %\end{algorithmic}
        \textbf{Solve($G,s$)}\\ 
        1-\hspace*{0.5em}$d[\hspace*{0.1cm}]$ $\leftarrow $ \textit{Dijkstra} $\left(G,s\right)$ \\  
        2-\hspace*{0.5em}$E (G^{'})$ $\leftarrow$ $\emptyset$ \\ 
        3-\hspace*{0.5em}$V(G')$ $\leftarrow$ $V(G)$\\ 
        4-\hspace*{0.5em} \textbf{\textit{For All }} $u \xrightarrow[]{e} v$  $\in$ $E(G)$\\ 
        5-\hspace*{3em} \textbf{\textit{If}}$d[u] + w(e) = d[v] $\\
        6-\hspace*{6em} $E(G^{'}) = E(G^{'})  \cup  e $ \\ 
        7-\hspace*{0.5em} $T \leftarrow  Prim(G^{'} , s)$\\
        8-\hspace*{0.5em} \textbf{return } $T$ 
    \end{algorithm}
    
    

    \section{Complejidad temporal}

    \noindent La complejidad temporal del algoritmo esta dada por la complejidad Temporal de algoritmo de Dijkstra 
    que es $O(|E|\log |V|)$ , mas la complejidad de encontrar el subgrafo que tiene solo las aristas que estan en al menos 
    un camino minimo de $s$ a algun otro v\'ertice del grafo $G$ , esto se hace en las lineas de la 4-8 , y vemos que el for se 
    ejecuta $|E|$ veces  y a\~nadir la arista que cuple la propiedad (linea 6 )es $O(1)$ por lo que El tiempo de ejecucion del algoritmo 
    de la linea 4-8 es $O(|E|)$ . En la linea 9 se llama al algoritmo de Prim (visto en conferencia) que tiene una complejidad temporal
    $O(|E|\log |V|)$ . Luego por la regla de la suma la complejidad temporal del algoritmo es  $O(|E|\log |V|)$
\end{document}
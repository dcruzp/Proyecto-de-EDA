\documentclass{article}


\usepackage[spanish,es-noshorthands]{babel}
\usepackage{tikz}
\usepackage{minted}
\usepackage[hidelinks]{hyperref} 
\usepackage{color}
\usepackage{amsmath}
\usepackage{algpseudocode}
\usepackage{algorithm}
\usepackage{makeidx} 

\begin{document}
    \title{\textcolor{blue}{\textbf{Problema 3 }}\\}
    \author{Daniel de la Cruz Prieto C211\\ David Orlando De Quesada Oliva C211\\Javier Dominguez C212} 
    \date{}
    \maketitle  

    \section{Descripci\'on del problema} 

    Sea $G = <V,E>$ un grafo no dirigido y conexo con pesos positivos en las aristas, y un v\'ertice $s \in V$ . Un \'arbol de 
    caminos de costo m\'inimos con ra\'iz en s es un subgrafo $T = <V^{'} , E^{'}> $ donde $V^{'} \subseteq  V $     y $E^{'} \subseteq  E $ tal que: 
    
    \begin{itemize}
        \item $V^{'}$ es el conjunto de v\'ertices alcanzables desde s en $G$
        \item $T$ es un \'arbol con ra\'iz en $s$
        \item Para todo $v \in V^{'}$ el \'unico camino simple de $s$ a $v$ en $T$ es un camino de costo m\'inimo de $s$ a $v$ en $G$ .
    \end{itemize}

    \noindent Dise\~ne un algoritmo que encuentre en el grafo G un \'arbol T de caminos de costo m\'inimo con ra\'iz en s tal que
    la suma de los pesos de las aristas de T sea lo menor posible. La complejidad temporal del algoritmo debe ser
    $O(|E|log|V|)$.


    \section{ Explicaci\'on de la soluci\'on }

    \noindent La soluci\'on al problema es aplicar el algoritmo de  Dijkstra visto en conferencias.
    Para encontrar un subgrafo $G^{'}$ que contenga todas las aristas que forman parte de algun camino de 
    costo minimo desde $s$ hasta cualquier otro v\'ertice del grafo $G$. Luego encontramos un arbol $T$ abarcador de 
    costo m\'inimo en $G^{'}$ , $T'$ es el arbol que estamos buscando 

    \section{Seudoc\'odigo del Algoritmo }

    \begin{algorithm}[H]
        \caption{Determinar el \'arbol de caminos m\'inimos desde $s$ de menor peso }
        \begin{algorithmic}[1] 
            \State $d[\hspace*{0.1cm}]$ $\leftarrow $ Dijkstra $\left(G,s\right)$   
            \State $E (G^{'})$ $\leftarrow$ $\emptyset$
            \State $V(G')$ $\leftarrow$ $V(G)$
            \ForAll {$u \xrightarrow[]{e} v$  $\in$ $E(G)$}
                \If {$d[u] + w(e) = d[v] $} 
                    \State $E(G^{'}) = E(G^{'})  \cup  e $ 
                \EndIf
            \EndFor
            \State $T \leftarrow  PrimAlgorithm (G^{'} , s)$
            \State return $T$ 
        \end{algorithmic}
    \end{algorithm}
    
    \section{Complejidad temporal}

    \noindent La complejidad temporal del algoritmo esta dada por la complejidad Temporal de algoritmo de Dijkstra 
    que es $O(|E|\log |V|)$ , mas la complejidad de encontrar el subgrafo que tiene solo las aristas que estan en al menos 
    un camino minimo de $s$ a algun otro v\'ertice del grafo $G$ , esto se hace en las lineas de la 4-8 , y vemos que el for se 
    ejecuta $|E|$ veces  y a\~nadir la arista que cuple la propiedad (linea 6 )es $O(1)$ por lo que El tiempo de ejecucion del algoritmo 
    de la linea 4-8 es $O(|E|)$ . En la linea 9 se llama al algoritmo de Prim (visto en conferencia) que tiene una complejidad temporal
    $O(|E|\log |V|)$ . Luego por la regla de la suma la complejidad temporal del algoritmo es  $O(|E|\log |V|)$

    \section{Demostraci\'on  de la correctitud del algoritmo}

\end{document}
\documentclass{article}
\usepackage{algpseudocode,algorithm,algorithmicx}
\usepackage[usenames,dvipsnames,svgnames,table]{xcolor}
\usepackage{hyperref}
\begin{document}
    \title{\textcolor{blue}{\textbf{Problema 1 }}\\}
    \author{Daniel de la Cruz Prieto C211\\ David Orlando De Quesada Oliva C211\\Javier Dominguez C212} 
    \date{}
    \maketitle  

    \section*{Orden del problema} 

    Sean n casas en un vecindario y m tuber\'ias. Las tuber\'ias de agua subterr\'aneas conectan estas casas. Cada
    tuberı\'ia tiene cierta direcci\'on (el agua puede fluir solo en esta direcci\'on y no al rev\'es) y di\'ametro (que caracteriza
    la cantidad m\'axima de agua que puede manejar).
    \\[10pt]
    \noindent Para cada casa, hay como m\'aximo una tuber\'ia entrando y como m\'aximo una tuber\'ia saliendo de ella. Se quiere
    instalar tanques y grifos en las casas. Para cada casa con una tuber\'ia de agua saliente y sin una tuber\'ia de agua
    entrante, se debe instalar un tanque de agua en esa casa. Por cada casa con una tuber\'ia de agua entrante y
    sin una tuber\'ia de agua saliente, se debe instalar un grifo de agua en esa casa. Cada casa tanque transportar\'a
    agua a todas las casas que tengan una secuencia de tuber\'ias desde el tanque hasta ella. En consecuencia, cada
    casa de grifo recibir\'a agua procedente de alguna casa de tanque.
    \\[10pt]
    \noindent Para evitar que las tuber\'ias exploten una semana después tambi\'en se debe considerar el di\'ametro de las tuber\'ias.
    La cantidad de agua que transporta cada tanque no debe exceder el di\'ametro de las tuber\'ias que conectan un
    tanque a su correspondiente grifo. Se quiere encontrar la cantidad m\'axima de agua que se puede transportar de
    forma segura desde cada tanque hasta su grifo correspondiente. Dise\~ne un algoritmo que devuelva la cantidad
    m\'axima de agua que se puede trasnportar de forma segura desde cada tanque hasta su grifo correspondiente.
    La complejidad temporal de su algoritmo debe ser de \textit{O(n + m)}.
    \\\\
    Complejidad Temporal:\\
    La complejidad de nuestro algoritmo se basa en la complejidad del DFS $O(n+m)$.Lo \'unico que le agregamos al DFS
    es ir actualizando  el m\'inimo de las capacidad de las aristas que participan en un camino desde un tanque hasta un 
    grifo lo cual no afecta la complejidad temporal del mismo. Luego hacemos un recorrido por cada v\'ertice v del array max
    para imprimir las casa que son tanque con la m\'axima capacidad correspondiente que se puede transportar de forma segura 
    desde el mismo hasta un grifo,lo cual tiene complejidad $O(n)$.Por lo cual mi algoritmo tiene complejidad $O(n+m)$.
    Pseudoc\'odigo:
    
    \begin{algorithm}
        \caption{Calcular la capacidad máxima de agua que se puede transportar de forma segura desde cada tanque hasta su grifo correspondiente}
        \textbf{Solve($G$)\\}
        1-\hspace*{1em}max=[] -array donde voy tener para cada v\'ertice en caso que sea tenga la capacida m\'axima a transportar\\
        1-\hspace*{1em}DFS-Visit($G,u,min,ficvertex)$\\ 
        2-\hspace*{2em}u $\leftarrow$ visited \\
        3-\hspace*{2em}$time=time+1$ \\  
        6-\hspace*{2em}for each v $\in$ $Adj[u]$\\
        7-\hspace*{3em}do if v not visited\\
        9-\hspace*{4em}$\pi[v] \leftarrow u$\\
        10-\hspace*{4em}do if $c(u,v)< min$\\
        9-\hspace*{4em}$min=c(u,v)$\\
        10-\hspace*{4em}DFS-Visit(G,v,min)\\
        10-\hspace*{2em}do $if pi[u]=ficvertex$\\
        11-\hspace*{3em}max[u]=min\\        
        15-\hspace*{2em}return\\\\
        27-\hspace*{1em}for i in max\\
        28-\hspace*{2em}if $max[i] != \infty $\\
        29-\hspace*{3em}print(i,pi[i])\\
        
        %Me parece que seria mejor si lo agrego a una lista y despues pregunto por lista.count para no modificar mucho del codigo
        
    \end{algorithm}
\end{document}